\documentclass[a4paper,11pt]{article}
%-----------------------------------------------------------
\usepackage[top=0.75in, bottom=0.75in, left=0.50in, right=0.85in]{geometry}
\usepackage{graphicx,wrapfig,lipsum}
\usepackage{url}
\usepackage{palatino}
\usepackage{comment}
\fontfamily{SansSerif}
\selectfont

\usepackage[T1]{fontenc}
\usepackage[ansinew]{inputenc}
% \usepackage{helvetica}
% \usepackage{array}
\usepackage{color}
\definecolor{mygrey}{gray}{0.65}
\textheight=10in
\raggedbottom
% \raggedright
\setlength{\tabcolsep}{0in}
\newcommand{\isep}{-2 pt}
\newcommand{\lsep}{-0.5cm}
\newcommand{\psep}{-0.6cm}
\renewcommand{\labelitemii}{$\circ$}
\pagestyle{empty}

\usepackage{makecell}
\usepackage{setspace}

%\usepackage{mathptmx}
%\usepackage[11pt]{moresize}
%-----------------------------------------------------------
%Custom commands
\newcommand{\resitem}[1]{\item #1 \vspace{-2pt}}
\newcommand{\resheading}[1]{{\small \colorbox{mygrey}{\begin{minipage}{0.975\textwidth}{\textbf{#1 \vphantom{p\^{E}}}}\end{minipage}}}}
\newcommand{\ressubheading}[3]{
\begin{tabular*}{6.62in}{l @{\extracolsep{\fill}} r}
	\textsc{{\textbf{#1}}} & \textsc{\textit{[#2]}} \\
\end{tabular*}\vspace{-8pt}}
% \textit{#3} &  \\
%-----------------------------------------------------------

\begin{document}

%\begin{comment}
\indent {\textbf{\Large  Sushmita Bhattacharya}}\hfill https://sushmitab.github.io/\\
\indent {Research Assistant}\hfill sushmita\_bhattacharya@g.harvard.edu\\
\indent {Harvard University}
\setlength{\tabcolsep}{5pt}

\vspace{10pt}

\resheading{\textbf{\large Education} }\\[\lsep]
\begin{description}
\item 
\begin{itemize}
\item \textbf{Harvard  University}\hfill Cambridge, MA, USA\\
Ph.D. in Computer Science\hfill \textit{July 2020 - Present}\\
Advisor: Dr. Stephanie Gil
\end{itemize}
\item
\begin{itemize}
\item \textbf{Arizona State University}\hfill Tempe, AZ, USA\\
	Ph.D. in Computer Science\hfill \textit{August 2018 - June 2020}\\
	Advisor: Dr. Stephanie Gil
\end{itemize}
\item
\begin{itemize}
\item \textbf{Indian Institute of Technology Bombay}\hfill Mumbai, India\\
	M.Tech. in Computer Science\hfill \textit{Fall 2013-Spring 2015}\\
	Advisor: Dr. N. L. Sarda
\end{itemize}
\item
\begin{itemize}
\item \textbf{Indian Institute of Engineering Science and Technology Shibpur}\hfill Howrah, India\\
	B.E. in Computer Science\hfill \textit{Fall 2007-Spring 2011}\\
	Advisor: Dr. Prasun Ghosal.
\end{itemize}
\end{description}

\resheading{\textbf{\large Research Interests}}
\begin{description}
	\item \hspace{0.75 cm} Reinforcement learning, Robotics, multi-agent systems, Machine learning, Deep learning.
\end{description}


%\hspace{0.5cm}
\resheading{\textbf{\large Publication}}
\begin{description}
	\item \hspace{0.65 cm} \textit{Reinforcement Learning for POMDP: Rollout and Policy Iteration with Application to Autonomous Sequential Repair Problems}, \underline{Sushmita Bhattacharya}, Thomas Wheeler, Stephanie Gil, and Dimitri Bertsekas, in IEEE Robotics and Automation Letters (RA-L), 2020 (10.1109/LRA.2020.2978451).
\end{description}

\resheading{\textbf{\large Awards}}
\begin{description}
	\item \hspace{0.65 cm} Engineering Graduate Fellowship from Ira A. Fulton Schools of Engineering (Spring 2020) for extraordinary academic achievements.
	
\end{description}

\resheading{\textbf{\large Research Project}}
\begin{description}
	\item 
	\begin{itemize}
		\item I am working on Reinforcement Learning for Partially Observable Markov Decision Processes. I look closely  at rollout and approximate policy iteration with the application to autonomous sequential repair problems.
	\end{itemize}
\item 
	\begin{itemize}
		\item My research involves scalable rollout algorithm for multiagent reinforcement learning especially in the context of POMDP applications over infinite horizon.
		%\item I am also working on various rollout based algorithm in context of search and rescue applications.
	\end{itemize}
%\item 
%\begin{itemize}
%	\item	We submitted our work "Reinforcement Learning for POMDP: Rollout and Policy Iteration with Application to Autonomous Sequential Repair Problems", \underline{Sushmita Bhattacharya}, Thomas Wheeler, Stephanie Gil, and Dimitri Bertsekas, in IEEE Robotics and Automation Letters (RA-L), 2020
%	\end{itemize}
\end{description}


\resheading{\textbf{\large Work Experience}}\\[\lsep]
\begin{description}
		\item 
	\begin{itemize}
		\item Research Assistant at Harvard University \hfill \textit{July 2020 - Present}
	\end{itemize}
	\item 
\begin{itemize}
	\item Graduate Research and Teaching Assistant at Arizona State University \hfill \textit{August 2018 - June 2020}
\end{itemize}
	\item 
	\begin{itemize}
		\item Software developer in Microsoft India Development Center. \hfill \textit{December 2016 - July 2018}
	\end{itemize}
	\item 
	\begin{itemize}
		\item Data Scientist in Honeywell Technology Solution Labs. \hfill \textit{July 2015 - December 2016}
	\end{itemize}
	\item 
	\begin{itemize}
		\item Teaching Assistant in Indian Institute of Technology Bombay \hfill \textit{July 2013 - June 2015}
	\end{itemize}
	\item 
	\begin{itemize}
		\item Developer in Cognizant Technology Solutions \hfill \textit{June 2011 - June 2013}
	\end{itemize}
\end{description}

\resheading{\textbf{\large Teaching Assistantship}}
\begin{description}
	\item
	\begin{itemize}
		\item CSE 691-Topics in Reinforcement Learning (Instructor: Dr. D. P. Bertsekas)\hfill\textit{ASU Spring 2020}
	\end{itemize}
	\item 
	\begin{itemize}
		\item CSE 591-Coordination of Multi-Robot Systems (Instructor: Dr. S Gil)\hfill  \textit{ASU Fall 2019}
	\end{itemize}
	\item 
	\begin{itemize}
		\item CSE 691-Topics in Reinforcement Learning (Instructor: Dr. D. P. Bertsekas)\hfill  \textit{ASU Spring 2019}
	\end{itemize}
	\item 
	\begin{itemize}
		\item CSE 471-Introduction to Artificial Intelligence (Instructor: Dr. S Gil)\hfill  \textit{ASU Spring 2019}
	\end{itemize}
	\item 
	\begin{itemize}
		\item CSE 574-Planning and Learning Methods in AI (Instructor: Dr. S Gil) \hfill \textit{ASU Fall 2018}
	\end{itemize}
	\item 
	\begin{itemize}
		\item CS 308 - Embedded Systems Lab (Instructor: Dr. Kavi Arya) \hfill \textit{IITB Spring, 2014}
	\end{itemize}
	\item 
	\begin{itemize}
		\item CS 387 - Database and Information Systems Lab(Instructor: Dr. N. L. Sarda) \hfill \textit{IITB Autumn 2014}
	\end{itemize}
	%Helped in data preparation work for project.	    
	%\begin{itemize}
	% 	\item Machine Learning lab\hfill \textit{(Autumn, 2013)}	    
	%    \end{itemize}
	%\end{itemize}
\end{description}

\resheading{\textbf{\large M.Tech. Project} }\\[\lsep]
\begin{description}
\item  \hspace{0.65 cm} 
%\begin{itemize}
 \textbf{Big Data Analysis in distributed streaming database}%\hfill \textit{(July 2014 - July 2015)}\\
\textit{(Guide: Prof. N. L. Sarda)}
      \begin{itemize}
	  %\item \textbf{Objective: }\textit{ATM Fraud Detection}
	  \item Storing large amount of transaction data in a reliable key-value store (HBase). Predicting customers spending habits using regression analysis using offline Hadoop map reduce jobs. Online detection of anomalous transactions using data mining techniques on sorted data using Apache Storm. Continuous integration of the outlier data with the existing HBase data store. 
	  \item Experimentation used Hadoop distributed file system (HDFS), Hadoop Map reduce techniques, HBase key-value data store, Apache Storm online streaming engine. Analyzed relative performance of traditional RDBMS databases and key value store HBase.
	    
 \end{itemize}
% \end{itemize}
%\item
%\begin{itemize}
%\item \textbf{Streaming Data Processing and Management}\hfill \textit{(Jan 2014 - May 2014)}\\
%\textit{(Guide: Prof. N. L. Sarda)}
%      \begin{itemize}
%	    \item Studied streaming data and its difference from traditional relational data and processing.
%	    \item Surveyed stream query language and special purpose storage and indexing for streams.
%	    \item Reviewed STREAM - a Stanford implementation for stream data management system.
%      \end{itemize}
%\end{itemize}

\end{description}


%\resheading{\textbf{\large Course Projects } }
%\begin{description}
%\item
%\begin{itemize}
%\item \textbf{Implementation of Table Partitioning in PostgreSQL}\hfill \textit{(Autumn, 2013)}\\
%\textit{(Guide: Prof. S. Sudarshan in Implementation Techniques for Relational Database Systems)} \hfill 
%\begin{itemize}
%  \item Modified source code of PostgreSQL to gain table (range) partitioning functionality
%  \item Changed code for insert, delete and update of tuples to take place in proper partition.
%  \item Added code for creating index(s) in the partitioned tables if one is present in main table.
%\end{itemize}
%\end{itemize}
%
%\item 
%\begin{itemize}
%\item \textbf{Color and Size Based Fruit Sorter using FireBird V}\hfill \textit{(Autumn, 2013)}\\
%\textit{(Guide: Prof. Kavi Arya and Prof. Krithi Ramamritham in Embedded and Real Time Systems)}
%\begin{itemize}
%\item Built modular hardware and software components for feature based, real time fruit sorter.
%\item Coded various sensors and actuators in Firebird V. Written code for serial communication between Firebird V and PC.
%\item Designed and performed experiments with various test-sets and got \textbf{90\% accuracy}. 
%\end{itemize}
%\end{itemize}
%
%
%\item 
%
%\begin{itemize}
%\item \textbf{Part of Speech Tagging} \hfill \textit{(Autumn, 2013)}\\
%\textit{(Guide: Prof. Pushpak Bhattacharya in Natural Language Processing)}
%\begin{itemize}
%\item Developed a part of speech tagging system for English sentences in \textbf{Java}, with an average precision of 93\%.
%\item Implemented using \textbf{Hidden Markov Model} and \textbf{Viterbi algorithm}.
%\end{itemize}
%\end{itemize}
%\item 
%
%
%\begin{itemize}
%\item \textbf{Understanding and Simulation of Network Performance in Dense Wifi Settings} \hfill \textit{(Spring, 2014)}\\
%\textit{(Guide: Prof. Mythili Vutukuru in Mobile Computing)} 
%\begin{itemize}
%\item Analyzed various statistics and parameters from network trace files in a wireless setup using \textbf{Python}.
%\item Simulated the same experiment using \textbf{NS3} simulation tool and tuned various network parameters to reflect the real experiment.
%%did a comparative study with the original experiment.
%\end{itemize}
%\end{itemize}
%\item
%
%\begin{itemize}
%\item \textbf{Geometry Generalization for Map Simplification} \hfill \textit{(Spring, 2014)}\\
%\textit{(Guide: Prof. N. L. Sarda in Spatial Database)} 
%\begin{itemize}
%\item Performed simplification of the linear geometries without affecting the topology of geometries using modified Ramer-Douglas-Peucker algorithm in \textbf{Java}.
%\item Achieved runtime of 300 ms to simplify a set of linestrings with 900 data points.
%\end{itemize}
%\end{itemize}
%
%\item
%
%\begin{itemize}
%\item \textbf{Exploited Vulnerabilities of Webview in Android} \hfill \textit{(Spring, 2014)}\\
%\textit{(Guide: Prof. Bernard Menezes in Cryptography and Network Security II)} 
%\begin{itemize}
%\item Designed attacks on Android Webview and analyzed methods to stop those attacks.
%\item Exploited vulnerabilities in calling Java code from Javascript and user interface through the developed Android Apps.
%\end{itemize}
%\end{itemize}
%
%
%\end{description}








%\resheading{\textbf{\large Skill Set}}\\
%
%\begin{tabular}{ll}
%	
%	\begin{minipage}{2in}
%		\begin{itemize}
%			\item \textit{Programming Languages:}
%		\end{itemize}
%	\end{minipage}  & C, C++, Core Java, PL/SQL \\
%	
%	\begin{minipage}{2in}
%		
%		\begin{itemize}
%			\item \textit{Scripting Languages:}
%		\end{itemize}
%	\end{minipage}  & Python, Bash\\
%	
%	\begin{minipage}{2in}
%		
%		\begin{itemize}
%			\item \textit{Operating Systems:}
%		\end{itemize}
%	\end{minipage}  & Linux, Windows \\
%	
%	
%	\begin{minipage}{2in}
%		\begin{itemize}
%			\item \textit{Tools:}
%		\end{itemize}  
%	\end{minipage}  & \LaTeX, SVN, Eclipse, Lex, Yacc, Android SDK, Make, NS3, Gnuplot \\
%	
%\end{tabular}
%\\[0.3cm]


\resheading{\textbf{\large Achievements \& Extra Curricular Activities }}

\begin{description}
\item
\begin{itemize}
	\item Secured All India Rank 57 among 2,24,160 candidates appeared in Graduate Aptitude Test in
Engineering, 2013 CSE.
\end{itemize}
\item
\begin{itemize}
\item Interests: painting, music.
\end{itemize}
\end{description}




\end{document}
