\documentclass[a4paper,11pt]{article}
%-----------------------------------------------------------
\usepackage[top=0.75in, bottom=0.75in, left=0.50in, right=0.85in]{geometry}
\usepackage{graphicx,wrapfig,lipsum}
\usepackage{url}
\usepackage{palatino}
\usepackage{comment}
\fontfamily{SansSerif}
\selectfont

\usepackage[T1]{fontenc}
\usepackage[ansinew]{inputenc}
% \usepackage{helvetica}
% \usepackage{array}
\usepackage{color}
\definecolor{mygrey}{gray}{0.65}
\textheight=10in
\raggedbottom
% \raggedright
\setlength{\tabcolsep}{0in}
\newcommand{\isep}{-2 pt}
\newcommand{\lsep}{-0.5cm}
\newcommand{\psep}{-0.6cm}
\renewcommand{\labelitemii}{$\circ$}
\pagestyle{empty}

\usepackage{makecell}
\usepackage{setspace}
\usepackage{textcomp}
%\usepackage{mathptmx}
%\usepackage[11pt]{moresize}
%-----------------------------------------------------------
%Custom commands
\newcommand{\resitem}[1]{\item #1 \vspace{-2pt}}
\newcommand{\resheading}[1]{{\small \colorbox{mygrey}{\begin{minipage}{0.975\textwidth}{\textbf{#1 \vphantom{p\^{E}}}}\end{minipage}}}}
\newcommand{\ressubheading}[3]{
	\begin{tabular*}{6.62in}{l @{\extracolsep{\fill}} r}
		\textsc{{\textbf{#1}}} & \textsc{\textit{[#2]}} \\
	\end{tabular*}\vspace{-8pt}}
% \textit{#3} &  \\
%-----------------------------------------------------------

\begin{document}
	
	%\begin{comment}
	\indent {\textbf{\Large  Sushmita Bhattacharya}}\hfill https://sushmitab.github.io/\\
	\indent {Ph.D. Candidate}\hfill sushmita\_bhattacharya@g.harvard.edu\\
	\indent {Harvard University}
	\setlength{\tabcolsep}{5pt}
	
	\vspace{10pt}
	
	\resheading{\textbf{\large Research Interests}}
	\begin{description}
		\item \hspace{0.75 cm} Reinforcement learning, Autonomous multiagent systems, Robotics, Machine learning, and Deep learning.
	\end{description}
	
	
	
	\resheading{\textbf{\large Education} }\\[\lsep]
	\begin{description}
		\item 
		\begin{itemize}
			\item \textbf{Ph.D. in Computer Science}\\
			\textbf{Harvard  University}, MA, USA\hfill\textit{July 2020 - Present}\\
			\textbf{Arizona State University} AZ, USA\hfill \textit{August 2018 - June 2020}\\
			Advisor: Dr. Stephanie Gil
		\end{itemize}
		\item
		\begin{itemize}
			\item \textbf{M.Tech. in Computer Science}\\
			\textbf{Indian Institute of Technology Bombay}, India\hfill\textit{July 2013-June 2015}\\
			Advisor: Dr. N. L. Sarda
		\end{itemize}
		\item
		\begin{itemize}
			\item\textbf{B.E. in Computer Science} \hfill\\
			\textbf{Indian Institute of Engineering Science and Technology Shibpur}, India
			\hfill \textit{July 2007-May 2011}\\
			Advisor: Dr. Prasun Ghosal.
		\end{itemize}
	\end{description}
	
	
	
	
	
	
	
	\resheading{\textbf{\large Research Projects}}
	\begin{description}
		\item 
		\begin{itemize}
			\item\textbf{Multiagent reinforcement learning for POMDP}\hfill %\textit{Ongoing work}
			\item Developed scalable multiagent rollout algorithms for large Partially Observable Markov Decision Processes (POMDP) with large state and control spaces. 
			The proposed policies demonstrated cost improvement property using approximate policy iteration with a scalable implementation.
			\item The proposed algorithm reduced computational complexity from an exponential to a linear in the number of agents. 
			The generated policies demonstrated coordinated behavior, making them suitable for POMDP applications with large teams of robots.
			\item Applied the algorithms to a class of cooperative multiagent dynamical sequential repair problems, and the results outperformed several state-of-art methods.
			\item The proposed methods worked well under imperfect agent communication, with local and intermittent cloud communication.
		\end{itemize}
		\item 
		\begin{itemize}
			\item\textbf{Reinforcement learning for POMDP}
			\item Developed online rollout algorithms for large Partially Observable Markov Decision Processes (POMDP) with large state spaces. Improved cost of the rollout policy using approximate policy iteration where successive policies were approximated using neural networks.
			\item Proposed feature-based partitioning of the state-space and used multiple neural networks to deal with exploration-exploitation issues by facilitating parallel computation.
			\item Applied the algorithms to a class of time-critical dynamical sequential repair problems, and the results outperformed several state-of-art methods.
			
		\end{itemize}
		
	\end{description}
	
	
	\resheading{\textbf{\large Work Experience}}\\[\lsep]
	\begin{description}
		\item 
		\begin{itemize}
			\item Research Assistant at Harvard University \hfill \textit{July 2020 - Present}
		\end{itemize}
		\item 
		\begin{itemize}
			\item Graduate Research and Teaching Assistant at Arizona State University \hfill \textit{August 2018 - June 2020}
		\end{itemize}
		\item 
		\begin{itemize}
			\item Software developer in Microsoft India Development Center. \hfill \textit{December 2016 - July 2018}
		\end{itemize}
		\item 
		\begin{itemize}
			\item Data Scientist in Honeywell Technology Solution Labs. \hfill \textit{July 2015 - December 2016}
		\end{itemize}
		\item 
		\begin{itemize}
			\item Teaching Assistant in Indian Institute of Technology Bombay \hfill \textit{July 2013 - June 2015}
		\end{itemize}
		\item 
		\begin{itemize}
			\item Developer in Cognizant Technology Solutions \hfill \textit{June 2011 - June 2013}
		\end{itemize}
	\end{description}
	
	\resheading{\textbf{\large Teaching Assistantships}}
	\begin{description}
		\item
		\begin{itemize}
			\item CSE 691-Topics in Reinforcement Learning (Instructor: Dr. D. P. Bertsekas)\hfill\textit{ASU Spring 2020}
		\end{itemize}
		\item 
		\begin{itemize}
			\item CSE 591-Coordination of Multi-Robot Systems (Instructor: Dr. S Gil)\hfill  \textit{ASU Fall 2019}
		\end{itemize}
		\item 
		\begin{itemize}
			\item CSE 691-Topics in Reinforcement Learning (Instructor: Dr. D. P. Bertsekas)\hfill  \textit{ASU Spring 2019}
		\end{itemize}
		\item 
		\begin{itemize}
			\item CSE 471-Introduction to Artificial Intelligence (Instructor: Dr. S Gil)\hfill  \textit{ASU Spring 2019}
		\end{itemize}
		\item 
		\begin{itemize}
			\item CSE 574-Planning and Learning Methods in AI (Instructor: Dr. S Gil) \hfill \textit{ASU Fall 2018}
		\end{itemize}
		\item 
		\begin{itemize}
			\item CS 308 - Embedded Systems Lab (Instructor: Dr. Kavi Arya) \hfill \textit{IITB Spring, 2014}
		\end{itemize}
		\item 
		\begin{itemize}
			\item CS 387 - Database and Information Systems Lab(Instructor: Dr. N. L. Sarda) \hfill \textit{IITB Autumn 2014}
		\end{itemize}
		
	\end{description}
	
	\resheading{\textbf{\large M.Tech. Project} }\\[\lsep]
	\begin{description}
		\item \hspace{0.65 cm} \textbf{Big Data Analysis in distributed streaming database}
		%\hfill\textit{(Guide: Prof. N. L. Sarda)}
		\begin{itemize}
			\item Developed an application for studying customer spending habits using regression analysis with offline Hadoop map-reduce jobs. The trend results were stored in a reliable HBase key-value store to facilitate online detection of anomalous transactions using data mining techniques and Apache Storm. 
			
			
		\end{itemize}
		
	\end{description}
	
	
	\resheading{\textbf{\large Awards}}
	\begin{description}
		\item \hspace{0.65 cm} Engineering Graduate Fellowship from Ira A. Fulton Schools of Engineering (Spring 2020) for extraordinary academic achievements.
		
	\end{description}
	
	\resheading{\textbf{\large Achievements \& Extra Curricular Activities }}
	
	\begin{description}
		\item
		\begin{itemize}
			\item Secured All India Rank 57 among 2,24,160 candidates appeared in Graduate Aptitude Test in Engineering, 2013 CSE.
		\end{itemize}
		\item
		\begin{itemize}
			\item Interests: painting, music.
		\end{itemize}
	\end{description}
	
	\resheading{\textbf{\large Publications}}
	\begin{description}
		\item %\hspace{0.65 cm}
		\begin{itemize}
			\item \textit{Reinforcement Learning for POMDP: Rollout and Policy Iteration with Application to Autonomous Sequential Repair Problems}, \textbf{Sushmita Bhattacharya}, Sahil Badyal, Thomas Wheeler, Stephanie Gil, and Dimitri Bertsekas, in IEEE Robotics and Automation Letters (RA-L), 2020. %(10.1109/LRA.2020.2978451).
		\end{itemize}
		\item 
		\begin{itemize}
			\item \textit{Multiagent Rollout and Policy Iteration for POMDP with Application to Multi-Robot Repair Problems}, \textbf{Sushmita Bhattacharya}, Siva Kailas, Sahil Badyal, Stephanie Gil, and Dimitri Bertsekas, in Conference on Robot Learning (CoRL), 2020. 
		\end{itemize}
	\end{description}
	
	
	\iffalse
	\resheading{\textbf{\large References}}
	\begin{description}
		\item \hspace{0.75 cm}\textbf{Dr. Stephanie Gil}\\School of Engineering and Applied Sciences (CS)\\Harvard University\\Email: sgil@seas.harvard.edu
		\item \hspace{0.75 cm}\textbf{Dr. Dimitri P. Bertsekas}\\McAfee Professor of Engineering\\Massachusetts Institute of Technology,\\Fulton Professor of Computational Decision Making\\Arizona State University\\Email: dimitrib@mit.edu
	\end{description}
	\fi
	
	
\end{document}
